\documentclass[12pt]{extarticle}

% Language setting
% Replace `english' with e.g. `spanish' to change the document language
\usepackage[english]{babel}
\usepackage[
    backend=biber,
    bibstyle=ieee,
    citestyle=numeric,
]{biblatex}
\addbibresource{cite.bib} %Imports bibliography file
\usepackage{soul}
\usepackage[export]{adjustbox}
\usepackage{listings}
\usepackage{makecell}
\usepackage{longtable}
\usepackage[table,x11names]{xcolor}
% Set page size and margins
% Replace `letterpaper' with`a4paper' for UK/EU standard size
\usepackage[letterpaper,top=2cm,bottom=2cm,left=3cm,right=3cm,marginparwidth=1.75cm]{geometry}

% Useful packages
\usepackage{amsmath}
\usepackage{graphicx}
\usepackage{subcaption}
\usepackage[colorlinks=true, allcolors=blue]{hyperref}
\usepackage{textcomp}

\title{
    Pretext Tasks: \\
    can they help against adversarial attacks?
}
\author{Artem Sereda artem.sereda@campus.tu-berlin.de}

\begin{document}
    \maketitle

    \begin{abstract}
    The continuous improvements in the image classification over the recent years are opening doors to
    a lot of life-changing applications of Machine Learning (ML).
    It has been proven that including a pretext task improves classifiers' standard accuracy.
    Despite numerous studies of an adversarial vulnerability phenomenon, there is no clear answer
    how a choice of pre-training influences Neural Networks' (NNs') adversarial robustness.
    This paper will first give a general introduction to white box adversarial attacks using Fast Gradient Sign Method (FGSM),
    as well as to some popular pretext tasks.
    Afterwards the impact of including a pretext task, as well as its choice will be evaluated.
    The implementation of the evaluation for this paper can be found in accompanying GitHub repository~\cite{github}.
    \end{abstract}


    \section{Introduction}

\paragraph{Motivation}
In recent years, Neural NNs have had a lot of success in image classification.
This has led to advances in many areas, including computer vision and object recognition.
E.g. advances in a constantly ongoing "ImageNet classification challenge", which were achieved
through new approaches,
rather than increasing NNs number of layers and parameters~\cite{russakovsky2015imagenet,DBLP:journals/corr/abs-1905-11946}.
However, NNs are also vulnerable to adversarial attacks~\cite{ilyas2019adversarial}.
Adversarial attacks are inputs designed to fool NN into miss-classifying data.
This can have serious consequences, as it can lead to incorrect results or decisions.
It is researchers' obligation to assure ML classifiers' trust-worthiness.

\paragraph{Adversarial attack}
Goodfellow defined adversarial attacks as “inputs to machine learning models that an
attacker has intentionally designed to cause the model to make a mistake.”
~\cite{DBLP:journals/corr/abs-1802-08195} \\
In the domain of image classification, adversarial attacks are usually formed by applying a small perturbation (which
is barely noticeable for human viewer) to a naturally occurring image, with the intention of making NN miss-classify.
There are many types of adversarial attacks.
This paper focuses on white-box attacks, where the attacker has full access to the model and its parameters,
namely FGSM.

\paragraph{FGSM}
This method was first introduced by Goodfellow and Jonathon Shlens and Christian Szegedy
~\cite{goodfellow2015explaining}.
It produces adversarial images which make NN miss-classify,
but are still recognisable as of the same class for human viewer.
An example of such images can be seen in~\ref{fig:fig-adv}.
FGSM works by using the gradients of the neural network to create an adversarial pattern.
For an input image,
the method evaluates the signed gradient of the loss function with respect to the input image to create a pattern,
which maximizes the loss.
The pattern is then added pixel wise to the original image.
The new image is called the adversarial image.
The process can be summarised using the following expression~\ref{eq:adv}.
\begin{equation}\label{eq:adv}
    adv\_x = x + \epsilon \cdot sign(\nabla_x J(\theta, x, y))
\end{equation}
where: \\
$adv_x$ is the resulting image, \\
$x$ is the original image, \\
$\epsilon$ is the intensity of adversarial pattern, \\
$J$ is the loss function, \\
$\theta$ is NNs parameters, \\
$y$ is the original label.
\\
\begin{figure}[h]
    \begin{subfigure}{0.4\textwidth}
        \caption{Tulips 40\% confidence}
        \includegraphics[width=8cm]{images/og_image}
    \end{subfigure}
    \begin{subfigure}{0.4\textwidth}
        \caption{Adversarial pattern for this image}
        \includegraphics[width=8cm]{images/adv_pattern}
    \end{subfigure}
    \\
    \begin{subfigure}{0.4\textwidth}
        \caption{$\epsilon = 0.01$, Roses 40\% confidence}
        \includegraphics[width=8cm]{images/adv_attack_001}
    \end{subfigure}
    \begin{subfigure}{0.4\textwidth}
        \caption{$\epsilon = 0.1$, Roses 40\% confidence}
        \includegraphics[width=8cm]{images/adv_attack_01}
    \end{subfigure}

    \caption{An example image, adversarial pattern created from it, and adversarial attack with different values of $\epsilon$}
    \label{fig:fig-adv}
\end{figure}



\paragraph{some meanigfull transition} bla bla bla
\\
\\
One should always want to train ML classifier, in such a way, which maximizes generalization.
Naive approach for this would be to provide as much diverse training data as possible.
This approach, however, heavily relies on a collection and labeling of data by humans, which is an extremely time-consuming

\paragraph{Transfer Learning}
Another popular technic aiming to improve ML classifiers accuracy is called Transfer Learning (TL).
TL is a training approach in ML, which focuses on storing knowledge gained while solving one
problem and applying it to a different but related problem.
For example, knowledge gained while learning to recognize cars could apply when trying to recognize trucks.
Concrete for ML engineers this could mean reusing one of the NNs, which performed the best at e.g.
"ImageNet classification challenge", and then fine-tuning it for a concrete downstream task of interest.
This eliminated the need of collecting and labeling more data for a concrete classification problem,
however, it still heavily relies on the existence of big labeled datasets.
Not to forget, that while NN reuses the learned features, the NN will also reuse
the biases learned from previous datasets, which makes them once again more dependable on quality of data
collected previously by a human.


\paragraph{Self-Supervised Learning Framework}
In contrast to previously described frameworks, the Self-Supervised Learning framework, requires relatively small amount
of data to produce a big variety of labeled training data.
One of its core features, is that
labeling of new data is not dependent on humans, and can be done in seconds by the ML classifier itself.
E.g. NN applies multiple pre-defined data augmentations to existing data and overwrites the labels by corresponding
pseudo-labels, and then is trained to predict which one was applied.
This allows multiplying the size of
training dataset by a number of pre-defined documentations, and also increasing the diversity of training data
(the metric to measure this, is out of the scope of this paper).

Normally, after self-supervised learning, similarly to TL fine-tuning would be executed.
These parts of training are called pre-text task (or training) and downstream task (or training) respectively.

\paragraph{Pretext task}
Pretext task is the self-supervised learning task solved to learn visual representations,
with the aim of using the learned representations or model weights obtained in the process, for the downstream task.
More concrete for image-classification, the weights of convolutional layers will be frozen after pre-text training,
with the aim of reusing the features learned, similarly to TL.

It has been shown that pretext tasks can significantly improve NNs accuracy~\cite{kolesnikov2019revisiting}.
It is also believed they contribute to NNs learning of important (as per human agents) features.
This paper focuses on TL, rotation and jigsaw pre-text tasks.


\paragraph{Related wörk}
The adversarial vulnerability phenomenon is widely known, first described by Ian J. Goodfellow~\cite{goodfellow2015explaining}.
It has been the accompanying success of image classification over the last years.
Up to day, there is no clear solution to this problem.
One of the suggested approaches is Adversarial Training~\cite{https://doi.org/10.48550/arxiv.1805.12152},
during which NN is trained to reduce the loss over worst-case adversarial perturbations.
This, however, comes at a cost of decreased standard classification accuracy, not the least this approach is noticeably more time-consuming
There have also been numerous studies~\cite{kolesnikov2019revisiting,DBLP:journals/corr/NorooziF16,DBLP:journals/corr/abs-1912-01991}
on how pretext tasks, and their choice can influence standard classification accuracy.
On the other hand, there were no studies, which investigated the impact of pretext task choice on
adversarial vulnerability. \\
This study was motivated by \st{curiosity} increasing demand in highly accurate robust trust-worthy ML classifiers.





    \section{Methods}

\paragraph{Model}
Convolutional networks' architectures for image recognition have evolved quite drastically in recent years,
with numerous options available "out of the box".
E.g.\ Efficient Net (EffNet) delivers impressive accuracy, while being able to scale better than a lot of
previous architectures~\cite{DBLP:journals/corr/abs-1905-11946}.
For this paper, the evaluation was done using it, namely EfficientNetB0~\cite{KerasEffNet}

\paragraph{Rotation pretext task}
A common choice of pretext task could be to produce 4 copies of
a single image by rotating it by {0°, 90°, 180°, 270°} and let a single network predict the rotation which was applied.
Alexander Kolesnikov and Xiaohua Zhai and Lucas Beyer: "Intuitively, a good model should learn to
recognize canonical orientations of objects in natural images" ~\cite{kolesnikov2019revisiting}.
For this papers evaluation, each image from the original dataset was rotated 0°, 90°, 180°,
270° and assigned a new pseudo label from [0, 1, 2, 3] accordingly.
The examples of such images are shown in~\ref{fig:rot-fig}.
All 4 batches of rotated images, as well as pseudo labels were concatenated in new dataset, and shuffled.
Then EffNet was then trained to identify rotation applied, afterwards the convolutional layers were frozen,
and the output layer size was adjusted for the downstream task \footnote{For details about pretext tasks implementations please refer to
impl/util/pretext.py in accompanying GitHub repository~\cite{github} \label{fn-pre}}.

\begin{figure}[h]
    \begin{subfigure}{0.33\textwidth}
        \caption{Label = 0}
        \includegraphics[width=5cm]{images/rot_0}
    \end{subfigure}
    \begin{subfigure}{0.2\textwidth}
        \caption{Label = 1}
        \includegraphics[width=5cm]{images/rot_1}
    \end{subfigure}
    \begin{subfigure}{0.33\textwidth}
        \caption{Label = 2}
        \includegraphics[width=5cm]{images/rot_2}
    \end{subfigure}
    \caption{An example how images used for rotation pretext task could look like}
    \label{fig:rot-fig}
\end{figure}


\paragraph{Jigsaw pretext task}
The task is
to recover relative spatial position of 4 sampled image patches
after a random permutation of these patches was performed
~\cite{kolesnikov2019revisiting}.
All of these patches are concatenated in 'puzzle' image,
which is later sent through the same network, which needs to predict a permutation applied.\
A similar approach as described by Mehdi Noroozi and Paolo Favaro~\cite{DBLP:journals/corr/NorooziF16} was adopted.
Random 4 out of 24 possible permutations were chosen for each batch
(number of possible permutations can be obtained from Newtonian binomial $P=\frac{r!}{(r-n)!}$).
For each permutation, each image was cut in 4 equal parts,
afterwards these tiles were permuted as per chosen permutation, and concatenated in 1 (puzzle) image.
An example can be seen in~\ref{fig:jig-fig}.
Similarly, to rotation pseudo labels in [0\ldots23] had been assigned,
NN was trained to identify permutation applied.
Then the weights of convolutional layers were frozen and reused for the downstream task ^{\ref{fn-pre}}.

\\
\begin{figure}[h]
    \begin{subfigure}{0.33\textwidth}
        \caption{Original image}
        \includegraphics[width=5cm]{images/dandelion}
    \end{subfigure}
    \begin{subfigure}{0.2\textwidth}
        \includegraphics[width=3cm]{images/arrow}
    \end{subfigure}
    \begin{subfigure}{0.33\textwidth}
        \caption{Generated puzzle, label=1}
        \includegraphics[width=5cm]{images/puzzle}
    \end{subfigure}
    \caption{An example how puzzle for jigsaw pretext task could look like}
    \label{fig:jig-fig}
\end{figure}


\paragraph{Transfer Learning}For this paper, EffNet was pre-trained on "imagenette"~\cite{ImageNette} dataset for image classification.
Imagenette is a subset of 10 easily classified classes from the Imagenet dataset.
Afterwards, the NN was fine-tuned for an actual classification task of interest on~\cite{tfflowers}.

\paragraph{Adversarial Images with FGSM}
The implementation of FGSM for this paper was based on~\cite{FGSM}.
In order to generate an adversarial pattern for each image, as per previously described formula~\ref{eq:adv}
gradient of loss function is evaluated with sign for each image.
Then as per previously described approach,
an adversarial pattern was added pixel wise to the original image with $\epsilon = 0.01$.
The resulting adversarial image was then clipped by value, so each channels' value stays in interval [0\ldots255],
as required for RGB color encoding
\footnote{For details about adversarial attack implementation please refer to
impl/util/adversarial.py in accompanying GitHub repository~\cite{github} \label{fn-adv}}.


\newpage
\paragraph{Evaluation approach}
In order to evaluate, how including pretext task in the training process influences NNs vulnerability against adversarial attacks,
I have measured miss-classification rate while keeping the intensity of adversarial pattern fixed at $\epsilon = 0.01$.
\\
Following metrics of interest were recorded:

\begin{equation}
    Accuracy_i = \frac{\# \; images \; correctly \; classified_i}{\# \; test \; images_i} \cdot 100 \%
\end{equation} \\
where Accuracy\_i is the accuracy in each evaluation round \\
\# stands for "number".

\begin{equation}
    Miss \; rate_i = \frac{\# images \; miss \; classified_i}{\# \; images \; correctly \; classified_i} \cdot 100 \%
\end{equation}
where Miss rate\_i is the miss rate in each evaluation round.


The dataset "tf\_flowers"~\cite{tfflowers} was chosen for experiment, 5\% of dataset were reserved for evaluation.
\\
During each evaluation round NN network was pre-trained either with transfer learning, or rotation, or jigsaw.
The number of training epochs was varied in [15, 30, 45, 60] for pretext training, while the number of epochs for downstream
training was fixed at 30.
\\
The reserved for evaluation images were given to NN. The number of correct classifications was recorded.
Then, for each previously correctly classified image FGSM~\ref{eq:adv} with $\epsilon = 0.01$ was applied.
The number of miss-classifications was recorded afterwards \footnote{For details about evaluation implementations please refer to
impl/util/evaluation.py in accompanying GitHub repository~\cite{github} \label{fn-eval}}.


    \section{Results}

\paragraph{Baseline results}
\begin{itemize}
    \item  $\overline{Accuracy} = 37.285  \%$
    \item  $\overline{Miss \; rate} = 99.097 \%$
\end{itemize}

\begin{table}[h]
    \begin{tabular}{|r|r|r|}
        \hline
        \# Pre-text epochs & \overline{Accuracy}, \% & \overline{Miss \; rate}, \% \\
        \hline
        15                 & 43.004                  & 100.0                       \\
        30                 & 43.533                  & 100.0                       \\
        45                 & 42.882                  & 100.0                       \\
        60                 & 42.882                  & 100.0                       \\
        \hline
    \end{tabular}
    \caption{\label{tab:table-01}Transfer learning results}
\end{table}

\begin{table}[h]
    \begin{tabular}{|r|r|r|}
        \hline
        \# Pre-text epochs & \overline{Accuracy}, \% & \overline{Miss \; rate}, \% \\
        \hline
        15                 & 44.312                  & 96.325                      \\
        30                 & 44.482                  & 95.380                      \\
        45                 & 44.751                  & 95.115                      \\
        60                 & 42.603                  & 96.527                      \\
        \hline
    \end{tabular}
    \caption{\label{tab:table-1}Jigsaw results}
\end{table}

\begin{table}[h]
    \begin{tabular}{|r|r|r|}
        \hline
        \# Pre-text epochs & \overline{Accuracy}, \% & \overline{Miss \; rate}, \% \\
        \hline
        15                 & 41.889                  & 99.067                      \\
        30                 & 43.865                  & 97.581                      \\
        45                 & 43.865                  & 95.547                      \\
        60                 & 43.290                  & 97.135                      \\
        \hline
    \end{tabular}
    \caption{\label{tab:table-2}Rotation results}
\end{table}


\begin{figure}
    \includegraphics{images/acc}
    \caption{\label{fig:figure-1}Accuracy comparison}
\end{figure}

\begin{figure}
    \includegraphics{images/miss_rate}
    \caption{\label{fig:figure-2}Miss rate comparison}
\end{figure}




    \section{Discussion}

\paragraph{Goal}In this paper, I set up a goal to investigate if including a pretext task in the training process could reduce NNs
vulnerability against adversarial attacks.
The phenomenon of adversarial examples was extensively studied~\cite{ilyas2019adversarial, DBLP:journals/corr/abs-1802-08195, goodfellow2015explaining},
as well as its relation with standard accuracy~\cite{https://doi.org/10.48550/arxiv.1805.12152}.
However, ss to the day, there have been no studies of how choice of pretext task influences adversarial vulnerability.
There are also numerous studies about the influence of pretext tasks on standard
accuracy~\cite{DBLP:journals/corr/abs-1912-01991, DBLP:journals/corr/NorooziF16, kolesnikov2019revisiting}.


\paragraph{Observations and Interpretation}
All 3 pre-training approaches have shown a slight improvement.
Accuracy was increased by all pre-text tasks, with jigsaw performing the best and being up to 8\% more accurate.
Transfer learning and no pre-training had shown no decrease in miss-classification rate.
Rotation and jigsaw were able to reduce miss-classification rate up to 5\%, starting up from 45 training epochs.
\\
FGSM evaluates loss function of NN,
so it would be natural to assume that higher accuracy pairs with higher adversarial vulnerability.
\\
TF Flowers consist of 3670 items.
During pretext training, 4 pseudo classes were generated,
which results in 14680 images, and \textbf{18350} images together with downstream training.
Imagenette consists of 13,394, which results in \textbf{17064} images together with downstream training.
So the improvements could partially be caused by bigger training dataset size.
\\
\ldots
\\
The experiment was done for a $\epsilon = 0.01$ (which are not even visible for human viewer).
And the improvement of 5\% is far from solving the problem of adversarial vulnerability.

\paragraph{Limitations}This study comes with a few limitations,
most of them coming from to time and resource limitation, as well as working standalone on the research question.
\\
The most important one is the fact that the evaluation was done only for one (relatively small) dataset,
therefore, the result could be bound to the biases existing in it.
\\
Another one is the fact, that hyperparameters of NN (learning-rate, loss function etc.)
were not fine-tuned, which leaves the question open whether the
results presented have a causal relation with the choice of pretext task, or is it just correlation,
and similar results could be achieved by better chosen hyperparameters for NN .

\paragraph{Future work}
Adversarial vulnerability has been around \st{always?}, coming in pair with all the successes in image classification.
As ML provides a lot of fascinating possibilities, its reliability and robustness are still an open question.
Researchers should explore new methods and approaches to ensure them.
Concrete for pretext-tasks, this research could be extended by evaluating it on different datasets, fine-tuning NNs
hyperparameters, investing more time in pre-training as well as using models with a bigger number of trainable parameters
a.e. \ EfficientNetB1, EfficientNetB2 \ldots EfficientNetB7.

\paragraph{Conclusion}
Pretext training was able to produce a little improvement in accuracy as well as robustness against adversarial attacks.
However, these successes are far solving the problem of NNs vulnerability to adversarial attacks.




    \printbibliography
\end{document}
